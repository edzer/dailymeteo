%"Statistical modelling of space-time variability"
%dr. Gerard B.M. Heuvelink, Wageningen University (45 mins)

\documentclass{article}
\usepackage[utf8]{inputenc}
\title{\bf Spatial and temporal support of meteorological observations and predictions}
\author{Edzer Pebesma \\ Institute for Geoinformatics \\ University of Münster }
\begin{document}
\maketitle

{\em Support} refers to the physical size of the area, volume,
and/or temporal duration of a measured or predicted data value.
Support of measurements is often related to the physical constraints:
we cannot directly observe the temperature of a square kilometer,
not even of an area of 100 m$^2$; rainfall measurements also usually
refer to devices with a catchment area of les than 1 m$^2$.

By choosing measurement sites carefully, we hope, by the idea of
{\em representativity}, that measured values carry more information
about their surroundings than when they were {\em not} chosen with
the same care. Representativity could reflect the notion that we
would like to be able to measure average values over larger areas,
as local extreme conditions are typically avoided.

Nevertheless, measured value and local or regional averages
will differ. Geostatistical theory allows for predicting linearly
aggregated (mean) values by regularizing (averaging) semivariances,
and by predicting non-linearly aggregated values by simulation. The
type of aggregation (function), the aggregation predicate (target
support), and the variability of the predictant all play a role here.

{\em Aggregation} is the process of deriving a single number from a
collection of numbers. The aggregation function may be simple such
as in the case of {\em mean} or {\em max}, it may also be complex,
e.g. computing catchment discharge from spatially distributed
precipitation values. The aggregation predicate is the
spatial area and/or temporal period over which aggregation takes
place. Aggregation may be useful to (i) match data that is collected
at a coarser support (ii) increase accuracy of predictions, and
(iii) smooth out local, or short-term variability.

When we want to aggregate over a continuous area but do not have
exhaustive (continuous) measurement data available for this area,
a {\em model} for the observation data is required to fill the area
with missing data in with predictions. Typical models are stationary
covariance models, as used in geostatistics. When, in these models,
we assume the mean function to depend on external variables with a
different support (e.g. derived from satellite imagery, or elevation
data), we introduce a bias that depends on the difference of the
external variable at the support we have it and that, at the support
that would match that of the primary observation data. We will
discuss where this bias comes from, and how it may be dealt with.


\end{document}
